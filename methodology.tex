\chapter{Methodology}
\label{chp:sample_chapter}

This work makes use of some machine learning methods. The topic of machine learning is very large, so this paragraph section the field of machine learning in which this work is located.

Supervised vs Unsupervised Learning. Short explanation about the differences of this two types of Models. The used Group LASSO Method is a supervised learning Model.....
The earlier used TCK Method for example is a unsupervised Model because.....

Advantages / Disadvantages of the two Models....

In the case of this problem set, we have the issue, that we have to treat time series. This means, that it is difficult to use classical machine learning approaches. As an example, we can not use cross validation to validate the training and test data, what makes the Problem more difficult.

Relatively small amount of Data, because it is expensive to generate data.

The data must be labeled so that the inputs can be assigned to the corresponding output, which is necessary when training the models.

%***********************************************************************
\section{Data Preprocessing}
\label{sec:Data_Preprocessing}

\subsection{Outlier Detection}
\label{sec:Outlier_Detection}

To get good results, the data needs to be preprocessed. The raw data does have outliers and other anomalies, that influences the quality of the machine learning model. The raw signal does have some error, which should be detected by a automate outlier detection algorithm.


\subsection{Normalization of the Data}
\label{sec:Normalization_of_the_data}


\section{ARX Models}
\label{sec:ARX_models}


In this introduction a few terms are defined, which will be used in the following. As explained in previous chapters, temperature data measured with sensors and the thermal errors measured by an on-machine measurement cycle are required.

The temperature sensors are referred to as input in the following and are abbreviated with the letter $u$. The thermal errors are called output and are abbreviated with the letter $y$. 

\subsection{Modelling of thermal errors}
\label{sec:modelling of errors}


 \begin{equation}
	y[n]+a_1y[n-1]+...+a_{n_a}y[n-n_a] = b_0 u[n]+b_1u[n-1]+...+b_{n_b} u[n-n_b]
	\label{eq:arx_eq}
\end{equation}



 \begin{equation}
	y[n] = -(a_1y[n-1]+...+a_{n_a}y[n-n_a]) + b_0 u[n]+b_1u[n-1]+...+b_{n_b} u[n-n_b]
	\label{eq:arx_eq_extendet}
\end{equation}


 \begin{equation}
 	\underline{\varphi}[n] = \begin{bmatrix}
		-y[n-1]...-y[n-n_a]  u[n]...u[n-n_b]
				\end{bmatrix}^T
	\label{eq:arx_eq_matix}
\end{equation}

 \begin{equation}
	\underline{\theta} =  \begin{bmatrix}
		a_1 & a_2 & ... & a_{n_a} & b_0 & ... & b_{n_b}
				\end{bmatrix}^T
	\label{eq:thetavec}
\end{equation}

By these transformations equation \ref{eq:arx_eq_extendet} can be transformed to equation \ref{eq:matrixresult}.


 \begin{equation}
			\hat{y}[n] = \underline{\varphi}^T[n] \underline{\theta}		
	\label{eq:matrixresult}
\end{equation}



 \begin{equation}
			\hat{y}[n|\theta] = \underline{\varphi}^T[n] \underline{\theta}		
	\label{eq:matrixresult_2}
\end{equation}



\subsection{The Least Squares Method}
\label{sec:lestsquaremethod}



 \begin{equation}
			\underline{Y} = \underline{\underline{\Phi}} \, \underline{\theta}		
	\label{eq:LS_basic}
\end{equation}

 \begin{equation}
			\underline{Y} =  \begin{bmatrix}
		y(1)\\
		\vdots \\
		y(N)
				\end{bmatrix}	; 
							\underline{\underline{\Phi}} =  \begin{bmatrix}
		\underline{\varphi}^T(1)\\
		\vdots \\
		\underline{\varphi}^T(N)
				\end{bmatrix}	
	\label{eq:Y_Phi_matrice}
\end{equation}



 \begin{equation}
			\hat{\underline{\theta}}_{LS} = \mathrm{arg \underset{\theta} min} \Vert (\underline{Y}-\underline{\underline{\Phi}} \, \underline{\theta}) \Vert^2_2 	
	\label{eq:LS_intro}
\end{equation}

 \begin{equation}
			\hat{\underline{\theta}}_{LS} = (\underline{\underline{\Phi}}^T\underline{\underline{\Phi}})^{-1}\underline{\underline{\Phi}}^T\underline{Y}	
	\label{eq:LS_intro_solved}
\end{equation}


 \begin{equation}
			\underline{\underline{W}}  =  \begin{bmatrix}
		w_{1,1} & 0 & \cdots & 0 \\
		0 & w_{2,2} & \ddots & \vdots \\
		\vdots & \ddots & \ddots & 0 \\
		0 & \cdots & 0 & w_{N,N} \\
				\end{bmatrix},
				w \in {0|1}
	\label{eq:weightmatrix}
\end{equation}



 \begin{equation}
			\hat{\underline{\theta}}_{LS} = \mathrm{arg \underset{\theta} min} \Vert \underline{\underline{W}}^{\frac{1}{2}}(\underline{Y}-\underline{\underline{\Phi}} \, \underline{\theta}) \Vert^2_2 	
	\label{eq:w_LS_intro}
\end{equation}

 \begin{equation}
	\hat{\underline{\theta}}_{LS} = (\underline{\underline{\Phi}}^T\underline{\underline{W}} \, \underline{\underline{\Phi}})^{-1}\underline{\underline{\Phi}}^T \underline{\underline{W}} \, \underline{Y}
	\label{eq:solution_w_LS}
\end{equation}


\section{Group LASSO Method}
\label{sec:Group_LASSO}



 \begin{equation}
	\hat{\underline{\beta}}_{GL} = \mathrm{arg \underset{\beta} min} \Vert  (\underline{Y} - \sum_{l=1}^L \underline{\underline{X_l}} \, \underline{\beta_l} \Vert^2_2 + \lambda \sum_{l=1}^L \sqrt{p_l} \Vert \underline{\beta_l} \Vert_2
	\label{eq:grouplasso_original}
\end{equation}



 \begin{equation}
	\hat{\underline{\theta}}_{GL} = \mathrm{arg \underset{\theta} min} \Vert \underline{\underline{W}}^{\frac{1}{2}} (\underline{Y}-\underline{\underline{\Phi}} \, \underline{\theta}) \Vert^2_2 + \lambda \sum_{m=1}^p \sqrt{n^m_b} \Vert \underline{\theta}^m_B \Vert_2
	\label{eq:grouplasso}
\end{equation}

\begin{figure}[!htb]
    \centering
    \includegraphics[width=1.0\linewidth]{Inkscape/LASSO_und_LS} %NOTE that no .pdf has to be written
    \caption[2D Example Least Square vs. Gruop LASSO]{2D Exemple of a Least Square/Group LASSO Problem. On the left hand side: Least Square Problem. On the right hand side: Extension to Group LASSO problem. The grey circles in the centre of the coordinate System show the constraint of the LASSO term $\Vert \theta^m_B \Vert_2$, which is a circle in 2D space. The ellipsis on the right hand side describe the least square problem. The influence of the parameter $\lambda$ is shown from zero to infinity.}
    \label{fig:2D_LASSO}
\end{figure}


\section{Adaptiv Group LASSO Method}
\label{sec:Adaptive_Group_LASSO}

The Group LASSO method tends to select too many inputs. This is partly due to the way the parameter $\lambda$ is chosen, which is a direct consequence of the information criterion used. 
On the other hand, however, this is due to the method itself, which tends to select too many inputs. For this reason, the method was further developed. The first to do so was Fan et al.\cite{Fan2006} who modified the LASSO penalty so that different amounts of shrinkage are allowed for the different regression coefficients. At the same time, Zou et al. \cite{Zou_2006} introduced the name adaptive LASSO for the Method. As with the development of the Group LASSO, which evolved from the LASSO, the adaptive LASSO method was further developed by Wang et al. \cite{Wang_2008} into the adaptive Group LASSO method, which is used in this thesis. The basic equation has the following form:

 \begin{equation}
	\hat{\underline{\beta}}_{GL} = \mathrm{arg \underset{\beta} min} \Vert  (\underline{Y} - \sum_{l=1}^L \underline{\underline{X_l}} \, \underline{\beta_l} \Vert^2_2 + \sum_{l=1}^L \lambda_l \sqrt{p_l} \Vert \underline{\beta_l} \Vert_2
	\label{eq:original_adaptive_grouplasso_1}
\end{equation}

The first part of the equation corresponds to the least square term, which is identical to the first term in equation \ref{eq:grouplasso_original}, where the Group LASSO method is introduced. The modification is made in the second term. The parameter $\lambda$, which is in equation \ref{eq:grouplasso_original} before the sum sign of the penalty term, moves into the sum. The subscript l shows that this is no longer a global parameter. The parameter is selected separately for each group, which makes it possible to adapt it to the different groups. This has the advantage that the parameter $\lambda$ does not have to be set for strongly varying groups, which in the end is always only a compromise. 

The question arises as to how this parameter $\lambda_l$ should be set. As applied in Machine Learning ofr, this can be done using Cross Validation (CV). However, this is computationally intensive and therefore relatively expensive.  Furthermore, CV is not applicable because the concrete problem is a time series.

 \begin{equation}
	\lambda_l = \lambda \Vert \tilde{\theta}_l \Vert^{-\gamma}
	\label{eq:lambda_parameter}
\end{equation}


 \begin{equation}
	\hat{\underline{\theta}}_{GL} = \mathrm{arg \underset{\theta} min} \Vert \underline{\underline{W}}^{\frac{1}{2}} (\underline{Y}-\underline{\underline{\Phi}} \, \underline{\theta}) \Vert^2_2 + \lambda \sum_{m=1}^p \Vert \tilde{\theta}_m \Vert^{-\gamma} \sqrt{n^m_b} \Vert \underline{\theta}^m_B \Vert_2
	\label{eq:adaptive_grouplasso_2}
\end{equation}


\section{Alternating Direction Method of Multipliers}
\label{sec:ADMM}




 \begin{equation}
	\hat{\underline{\theta}}_{GL} = \mathrm{arg \underset{\theta} min} \Vert \underline{\underline{W}}^{\frac{1}{2}} (\underline{Y}-\underline{\underline{\Phi}} \, \underline{\theta}) \Vert^2_2 + \lambda \sum_{m=1}^p \sqrt{n^m_b} \Vert \underline{\theta}^m_B \Vert_2
	\label{eq:admm_gl}
\end{equation}


 \begin{equation}
	\mathrm{arg \underset{\theta, \varphi} min} \Vert \underline{\underline{W}}^{\frac{1}{2}} (\underline{Y}-\underline{\underline{\Phi}} \, \underline{\theta}) \Vert^2_2 + \lambda \sum_{m=1}^p \sqrt{n^m_b} \Vert \underline{\varphi}^m_B \Vert_2 \hspace{0.5cm} \mathrm{subject \: to} \hspace{0.5cm} \underline{\theta} - \underline{\varphi} = \underline{0}
	\label{eq:admm_extended}
\end{equation}




\begin{align}
	\underline{\theta}^{(k)} &= (\underline{\underline{\Phi}}^T \, \underline{\underline{W}} \, \underline{\underline{\Phi}} + \rho \underline{\underline{I}})^{-1}(\underline{\underline{\Phi}}^T \, \underline{\underline{W}} \, \underline{Y} + \rho (\underline{\varphi}^{(k-1)}-\underline{v}^{(k-1)})) \label{eq:eq1} \\ 
	\underline{\varphi}^{(k)}_g &= R_{c_g \lambda/ \rho}(\underline{\theta}_g^{(k)}+\underline{v}_g^{(k-1)}), \hspace{0.5cm} g= 1,...G  \label{eq:eq2} \\ 
	\underline{v}^{(k)} &= \underline{v}^{(k-1)}+ \underline{\theta}^{(k)}-\underline{\varphi}^{(k)}
	\label{eq:eq3}
\end{align}


 \begin{equation}
	R_{c_g \lambda/ \rho}(\underline{\vartheta}_g) =\biggl( 1-\frac{t}{\Vert \underline{\vartheta}_g \Vert_2}\biggr)_+  \cdot\underline{\vartheta}_g \hspace{1.5cm} t=\frac{\lambda}{\rho} \hspace{1.5cm}
	\label{eq:thresholdop}
\end{equation}




\begin{figure}[!htb]
    \centering
    \includegraphics[width=0.75\linewidth]{Inkscape/LASSO_exemple} %NOTE that no .pdf has to be written
    \caption[Example of 2D Group LASSO Problem]{This example shows how the ADMM method works in principle. If the amount of a parameter (in this case $\underline{\theta}_1$) falls below one, the soft-thresholding operator $R_t$ becomes negative and the parameter is set to zero (t=1). Nice to see in the plot that the parameter $\underline{\theta}_1$ is set to zero when the value falls below this value. 
}
    \label{fig:LASSO_exemple}
\end{figure}



\section{Information Criterion}
\label{sec:Information_Criterion}

The choice of the information criterion is crucial for the quality of the compensation, as well as the stability of the algorithm. In the literature there is a large selection of information criteria, which all have their own peculiarities and favor or penalize certain properties. The methodology used showed a behavior that tends to select too many inputs. Thus, an information criterion is needed that strongly penalizes many inputs. 

The algorithm was tested using three different information criteria. These are the Akaike Information Criterion (AIC), the Bayes Information Criterion (BIC), and the Pham Information Criterion (PIC). The formula for these three information criteria are listed below.


 \begin{equation}
	AIC = 2 \cdot k-ln(\hat{L})
	\label{eq:aic_general}
\end{equation}
	


 \begin{equation}
	AIC = 2 \cdot k+n \cdot ln(RSS)
	\label{eq:aic_rms}
\end{equation}


 \begin{equation}
	AIC = 2 \cdot k \cdot \sigma + n \cdot ln(RSS)
	\label{eq:aic_rms_adapt}
\end{equation}

\begin{figure}[!htb]
    \centering
    \includegraphics[width=1.0\linewidth]{figures/Model_complexity} %NOTE that no .pdf has to be written
    \caption[2D Example Least Square vs. Gruop LASSO]{Model Complexity}
    \label{fig:model_complexity}
\end{figure}




\section{Order Optimization}
\label{sec:order_opt}

Zero Order Model as generic algorithms are practical not used, because, they need to much computation time (computationally expensive)

When you use a first order Model as stochastic gradient descent, you just can find local optima, but in applications, this is good enough.

It's a trade off between the accuracy and the cost of your algorithm.

%***********************************************************************
\subsection{Optimization by Assuming Maximum Order}
\label{sec:optimization_may_order}



\subsection{Discrete Particle Swarm Optimization}
\label{sec:optimization}

Plot of the Searching space as an 3D example. (Graphik mit bereich[1E-6 5] [0 5] [0 5] wo eine Lösung gesucht wird. Lmabda, na und nb. Daran sollte zu erkennen sein, dass nicht alle Dimensionen Gleich sind. Zeigen, dass Lambda Kontinuierlich ist, na und nb jedoch disket)

\begin{figure}[!htb]
    \centering
    \includegraphics[width=1.0\linewidth]{figures/search_space} %NOTE that no .pdf has to be written
    \caption[Scheme of particle swarm optimization]{Searching space}
    \label{fig:flowchart}
\end{figure}


\begin{enumerate}
\item Initialize a population array of particles with random positions and velocities on D dimensions in the search space
\item \textbf{loop}

\begin{enumerate}

\item  For each particle, evaluate the desired optimization fitness function (introduced in section \ref{sec:opt_runtime}) in D variables
\item  Compare particle's fitness evaluation with its personal best $pbest_i$. If current value is better than \textit{$pbest_i$}, then set $pbest_i$ equal to the current value, and $p_i$ equal to the current location $x_i$ in D-dimensional space.
\item Identify the particle in the neighbourhood with the best success so far, and assign its index to the variable $g$.
\item Change the velocity and position of the particle according to the fallowing equation:

 \begin{equation}
 \hspace{-2cm}
		\begin{cases} \underline{v}_{i} \longleftarrow \omega \underline{v}_i +c_k \underline{r}_1 (p_{best} -p_i)+c_s \underline{r}_2(g_{best}-p_i) \hspace{1cm} \underline{r}_1,\underline{r}_2 \in (0,1) \\
		 x_{i} \longleftarrow x_i +v_{i} \end{cases}
	\label{eq:pso_pseudo}
\end{equation}

\item If a criterion is met (usually a sufficiently good fitness or a maximum number of iterations), exit loop.

\end{enumerate}

\item \textbf{end loop}
\end{enumerate}

\begin{figure}[!htb]
    \centering
    \includegraphics[width=1.0\linewidth]{Inkscape/PSO} %NOTE that no .pdf has to be written
    \caption[Scheme of particle swarm optimization]{Schematic representation of the PSO method. The two figures differ in the parameters $\omega$, $\underline{r}_1$, $\underline{r}_2$, $c_k$ and $c_s$ . The green or blue area shows the area in which the share of the global or personal speed vector can be. From the given data of the previous iteration and the random components, the new position then results, which is therefore subject to a certain randomness.}
    \label{fig:flowchart}
\end{figure}


 \begin{equation}
	\omega = \chi \hspace{1.5cm} c_k = \chi \phi_1 \hspace{1.5cm} c_s = \chi \phi_2 \hspace{1.5cm} \phi = \phi_1 + \phi_2
	\label{eq:c_parameter}
\end{equation}

 \begin{equation}
	\chi = \dfrac{2}{2-\phi-\sqrt{\phi^2-4\pi}}
	\label{eq:chi_parameter}
\end{equation}


%***********************************************************************

\section{Group LASSO for Thermal Error Compensation}
\label{sec:algorithm}




%-------------------------------



 


