\chapter{Results}
\label{chp:results}

This chapter presents the results of the adaptive input selection using the developed Group Lasso method. The first step is to introduce the experimental set-up. In a second step, the results of the Group LASSO method are explained and finally the Group LASSO method is compared with the existing TCK method so that the compensation results of the method can be evaluated.


%***********************************
\section{Machine Setup and Sensors}
\label{sec:machine_setup}



V [w C2' A' X' b Y Z C1 t].

\begin{figure}[!htb]
    \centering
    \includegraphics[width=0.8\linewidth]{Inkscape/Skizze_maschine} %NOTE that no .pdf has to be written
    \caption[Design of the machine]{Design of the MT used.}
    \label{fig:Design}
\end{figure}

The temperatures are measured within and around the MT. There are 8 sensors in the environment of the MT and 17 sensors measuring the temperature of machine components. The measured data and all sensors used are listed in detail in the appendix.

\section{Experimental Setup}
\label{sec:experimentalsetup}


\begin{table}[!htb]
\centering
\begin{threeparttable}
\caption[Experiment parameters]{Predefined parameters of the TALC for the experiment with a duration of 108 h (CP: calibration phase, NG: MT out of precision)}
\begin{tabular}{l l}
\hline
Parameter \hspace{7cm} & Value\\
  \hline
 Calibration phase & 12 h\\
 Measurement interval & 5 min\\
 Measurement interval (post CP) & 60 min\\
 Measurement interval (NG) & 5 min\\
 Action Control limits (ACL) & 5 $\mu m$ resp. 10 $\mu m/m$\\
 NG mode duration & 24 measurements\\
 Parameter update interval & 12 h\\
 Measurement cycle duration & 85 s\\
 Max. number of inputs per model & variable\\
\hline
\end{tabular}
\label{Tab:Experimental_setup}
\end{threeparttable}
\end{table}





\begin{figure}[!htb]
    \centering
    \includegraphics[width=0.8\linewidth]{results/speedprofil} %NOTE that no .pdf has to be written
    \caption[Speed profile]{Randomly generated speed profile of the C-axis. This runs on the MT during the test period of 108 hours.}
    \label{fig:speedprofil}
\end{figure}



\begin{figure}[!htb]
    \centering
    \includegraphics[width=0.5\linewidth]{Inkscape/Messzyklus} %NOTE that no .pdf has to be written
    \caption[Measuring cycle]{Used measuring cycle. The position of the sphere is measured in each specific axis position. Abbildung gemäss Blaser et al. \cite{Blaser_2017}}
    \label{fig:messzyklus.}
\end{figure}


\subsection{Comparison Between Static and Adaptive Sensor Set}
\label{sec:comparison_static_adaptive}


\begin{table}[!htb]
\centering
\begin{threeparttable}
%\captionsetup {width = 10cm}
\caption[Statistical evaluation of the data and comparison of the methods - static vs. adaptive]{Thermal position and orientation errors obtained with the Group LASSO Method.  Nomenclature from \cite{Blaser_2017}, for the investigated C-axis. The negative sign in the category peak value reduction means there is an increase of the thermal error due to the compensation.}
\begin{tabular}{c c c c c c c}
\hline
Error  & \multicolumn{2}{l}{Peak value} \hspace{2cm} & \multicolumn{2}{l}{Root mean square} \hspace{2cm} & \multicolumn{2}{l}{$\mathrm{95^{th}}$ percentile} \vspace{-0.2cm} \\
 & \multicolumn{2}{l}{reduction [\%]} & \multicolumn{2}{l}{error $\mu m / (\mu m / m)$} & \multicolumn{2}{l}{$\mu m / (\mu m / m)$} \\
\hline
  & static & adaptive & static & adaptive & static & adaptive \\
  \hline \vspace{-0.2cm}
 $\mathrm{E_{X0C}}$ & 74 & 76 & 1.3 & 1.0 & 1.7 & 1.4 \\ \vspace{-0.2cm}
 $\mathrm{E_{Y0C}}$ & 55 & 41 & 0.3 & 0.4 & 0.6 & 0.8 \\ \vspace{-0.2cm}
 $\mathrm{E_{R0T}}$ & 95 & 93 & 0.9 & 1.0 & 1.5 & 1.8 \\ \vspace{-0.2cm}
 $\mathrm{E_{Z0T}}$ & 94 & 94 & 0.3 & 0.3 & 0.5 &  0.5\\ \vspace{-0.2cm}
 $\mathrm{E_{A0C}}$ & 16 & 13 & 1.0 & 1.1 & 1.5 & 1.6 \\ \vspace{-0.2cm}
 $\mathrm{E_{B0C}}$ & 80 & 78 & 1.4 & 1.5 & 2.7 & 2.9 \\ 
 $\mathrm{E_{C0C}}$ & 56 & 58 & 2.4 & 2.6 & 1.7 & 2.1 \\
\hline
\end{tabular}
\label{Tab:statistic_GL}
\end{threeparttable}
\end{table}


\section{Comparison to the TCK method}
\label{sec:comparisonTCK}

\subsection{Graphical User Interface (GUI)}
\label{sec:graphical_user_interface}

\