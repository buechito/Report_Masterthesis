\chapter{Thermal Excitation}
\label{chp:thermal_excitation}

Putz et al. \cite{Putz2018}, \cite{Putz2018a}

Thermal Errors are measured at a wide variety of locations throughout the machine. 

(Picture of the machine with all sensors)

Explanation of the thermal errors. Explanation of what the influence of the cutting fluid is and how it can be recognized. As soon as the cutting fluid is switched off, undercooling takes place. This is due to evaporation, which extracts heat from the material. This undercooling can be seen in the plots by the kink that can be seen after the cutting fluid is switched off.

\begin{figure}[!htb]
    \centering
    \includegraphics[width=1.0\linewidth]{figures/thermische_einflusse} %NOTE that no .pdf has to be written
    \caption[2D Example Least Square vs. Gruop LASSO]{Thermal influence}
    \label{fig:thermische Einflusse}
\end{figure}


%***********************************************************************


\section{Heat Sources}
\label{sec:heat_sources}

The heat sources can be divided into external and internal sources. Therefore, a distinction is also made between external and internal excitation.


\section{Methods for External Excitation}
\label{sec:External_Excitation}

External environmental influences can be reproduced via different mechanisms. Two properties are problematic here. On the one hand, there is the question of how the excitation propagates in the machine, and on the other hand, in which time period the conditions change.

To simulate the latter, it is necessary to place the complete machine tool in a climate chamber. This allows the fastest possible temperature jumps to be simulated, which is important for a wide range of possible load cases.

\subsection{Open External Excitation}
\label{sec:Open_External_Excitation}

\subsection{Climate Chamber}
\label{sec:Climate_Chamber}

\section{Methods for Internal Excitation}
\label{sec:Internal_Excitation}

How can internal influences be replicated?
 

\subsection{One Dimensional Excitation}
\label{sec:1D_stimulation}



\subsection{Multidimensional Excitation}
\label{sec:multideimensional_stimulation}

\section{Combined Excitations}
\label{sec:combined_excitations}

Plot with order of different stimulations. The different stimulation profiles can be combined randomly and the feed of the different axis follow also a random profile. This combination simulates a load case, which can be compared with a real situation load case in a production environment.


%-------------------------------



 


