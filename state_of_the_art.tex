\chapter{State of the Art}
\label{chp:stateoftheart}

This chapter gives an overview of the methods currently used. First of all, the topic of thermal errors is introduced, which form the basis for this thesis. In the next step the TALC method is introduced, that is also used in this thesis. Further the adaptive input selection is described. Finally, the disadvantages of the previous methods and it illustrates why alternatives are needed.

%***********************************************************************
\section{Thermal Errors}
\label{sec:thermalerrors}

Short introduction about thermal Errors. What size do they have? They dnt't need  to be measured as a temperature. The signal can be a temperature, a humidity, or the current of a motor. So the question arises, if all this dimensions have the same magnitude. Normally this is not the case, which a normalization makes necessary. For the actual data, the z-score transformation shows to be the optimal one.



%***********************************************************************



\section{Thermal Error Compensation}
\label{sec:neuronal_networks}

\subsection{Machine Integration}
\label{sec:Machine_Integration}

Workspace position, one axis, all axis, complete machine integration

%-----------------------------------------------------
\subsection{Automation Level}
\label{sec:Automation_Level}


%-----------------------------------------------------
\section{Input Selection}
\label{sec:feature_selectiom}



%-----------------------------------------------------
\subsection{Adaptive Input Selection}
\label{sec:Adaptiveinputselection}

<%-----------------------------------------------------
\subsection{Times Series Cluster Kernel Method}
\label{sec:introTCK}



%-----------------------------------------------------
\section{Research Gap}
\label{sec:researchgap}

%-----------------------------------------------------
\section{Outline} %Gliederung
\label{sec:outline}

