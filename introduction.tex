% set counter to n-1:
\setcounter{chapter}{0}

\chapter{Introduction}
\label{sec:introduction}

This master thesis deals with the thermal behaviour of machine tools (MT). The methods used are from sub fields of machine learning and the models applied are data-driven. The work deals on the one hand with the selection of optimal model inputs and on the other hand with the generation of the best possible data for training the models. In order to make the methods understandable for the user, the code that runs in the back end is presented in the front end by means of a graphical user interface (GUI). 

%-----------------------------------------------------
\section{Motivation}
\label{sec:motivation}

The largest proportion of machine tool errors is due to thermal causes.

The machine tool is a complex thermal system.

For this reason, model-based compensation is used, which leads to higher dimensional systems that have to be optimized.


The evolution of the machine tool can be divided into four epochs. It all started with the Manuel operated machine tool. This consisted of a spindle drive and axes, which had to be operated by the operator via hand cranks. Compensation for thermal errors was not yet possible. The accuracy of the workpieces depended heavily on the operator, who needed a lot of feeling to produce accurate parts. 

The Great Revolution was the introduction of the numerically controlled machine tool (NC-MT). This was made possible by developments in computer technology, which allows the axes to be controlled individually. The MT's must be equipped with drives on the axes so that they can be moved automatically. These drives are controlled via the NC control, which allows the axes to be interpolated with each other. At the time of these developments, compensation for thermal errors was still out of the question. However, it became possible to compensate the tools. This meant that the operator no longer had to program the contour of his TCP, but could directly convert the contour of the workpiece. 


Show parallel development of computer technology and machine tools. The machine tool was created long before the first mechanical calculating machines were built. The possibilities for automatic control or monitoring were therefore out of the question for a long time. 

Plot with the two developments

1935 First mechanical calculating machine
1945 Relay based calculating machines
1950 Electron tube calculator
1955 Discovery of the transistor
1965 First integrated circuit (IC)
1970 First microchip
1980 First mention of the term Central Processing Unit (CPU)
2003 First multi-core processors
2015 Specialized chips for ML and AI tasks

The application of the latest computer technologies always lags a little behind the development of the machine tool. This is clear because these developments are aimed at the consumer market and must therefore first be adapted to the market of machine tool controls. 


\begin{figure}[!htb]
    \centering
    \includegraphics[width=0.9\linewidth]{Inkscape/wirkkette_2} %NOTE that no .pdf has to be written
    \caption[Economic and technical relationships]{Profitability depends on productivity, energy consumption and accuracy \cite{Weber2015}.}
    \label{fig:prf_accuracy}
\end{figure}
 
Möhring et al. \cite{Moehring2020}

Chen et al. \cite{Chen2019}

Liu et al. \cite{Liu2018}
