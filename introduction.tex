% set counter to n-1:
\setcounter{chapter}{0}

\chapter{Introduction}
\label{sec:introduction}

This chapter introduce the topic of the thesis.

%-----------------------------------------------------
\section{Motivation}
\label{sec:motivation}

The largest proportion of machine tool errors is due to thermal causes.

The machine tool is a complex thermal system.

For this reason, model-based compensation is used, which leads to higher dimensional systems that have to be optimized.

\subsection{Evolution of MT}
\label{sec:evolution}

The evolution of the machine tool can be divided into four epochs. It all started with the Manuel operated machine tool. This consisted of a spindle drive and axes, which had to be operated by the operator via hand cranks. Compensation for thermal errors was not yet possible. The accuracy of the workpieces depended heavily on the operator, who needed a lot of feeling to produce accurate parts. 

The Great Revolution was the introduction of the numerically controlled machine tool (NC-MT). This was made possible by developments in computer technology, which allows the axes to be controlled individually. The MT's must be equipped with drives on the axes so that they can be moved automatically. These drives are controlled via the NC control, which allows the axes to be interpolated with each other. At the time of these developments, compensation for thermal errors was still out of the question. However, it became possible to compensate the tools. This meant that the operator no longer had to program the contour of his TCP, but could directly convert the contour of the workpiece. 


\begin{figure}[!htb]
    \centering
    \includegraphics[width=0.9\linewidth]{Inkscape/wirkkette_2} %NOTE that no .pdf has to be written
    \caption[Economic and technical relationships]{Profitability depends on productivity, energy consumption and accuracy \cite{Weber2015}.}
    \label{fig:prf_accuracy}
\end{figure}
 



%-----------------------------------------------------
\section{Outline} %Gliederung
\label{sec:outline}

